\documentclass[11pt]{article}

\usepackage[normalem]{ulem}
\useunder{\uline}{\ul}{}
\usepackage{setspace}
\usepackage{abstract}
\usepackage[T1]{fontenc}
\usepackage[top=2cm, bottom=2cm, left=2cm, right=2cm]{geometry}
\usepackage{babel}
\usepackage{titling}
\usepackage{blindtext}
\usepackage{amsmath}
\usepackage{amsfonts}
\usepackage{multicol}
\usepackage{fancyvrb}
\usepackage{fancyhdr}
\usepackage{enumitem}
\usepackage{graphicx}
\usepackage{hyperref}
\renewcommand{\labelitemi}{$-$}
\setlength{\headheight}{13.6pt}
%\setlist[itemize]{itemsep=0pt}
\author{}

\begin{document}


\pagestyle{fancy}
\fancyhead{}
\fancyhead[L]{TFE - Création d'une application pour l'organisation des répétitions de spectacles.}
\fancyhead[R]{D. van Rossum}


\title{\vspace{-1cm}\huge{Test utilisateurs}\vspace{-1.7cm}}
\date{}
\maketitle
\thispagestyle{fancy}
\section{Créer un compte et s'identifier}
\begin{enumerate}
    \item Créer un compte avec comme nom \texttt{Laura} et avec comme nom de famille \texttt{Du Pont}.
    \item Mettre l'email \texttt{laura@mail.com}.
    \item Et comme mot de passe \texttt{laura123}.
    \item Ces professions son: danseur, chanteur et directeur artistique.
\end{enumerate}
\section{Taches en tant qu'organisateur}
\subsection{Créer un projet}
\begin{enumerate}
    \item Créer un projet avec comme nom \texttt{Aladin}.
    \item Le projet commence le \texttt{4 Mai} et termine le \texttt{4 Juin}.
    \item Ne pas mettre de description pour le moment.
\end{enumerate}
\subsection{Modifier le projet}
\begin{enumerate}
    \item Ajouter la description: \texttt{Spectacle de danse et de chant}
\end{enumerate}
\subsection{Ajouter des personnes au projet}
\begin{enumerate}
    \item Ajouter l'utilisateur avec comme email \texttt{jean@mail.com} avec les rôles: \texttt{Artiste de cirque} et \texttt{Comédien} au projet.
    \item Ajouter l'utilisateur avec comme email \texttt{eve@mail.com} et rôle: \texttt{Organisateur} au projet.
    \item Ajouter l'utilisateur avec comme email \texttt{victoria@mail.com} et rôle: \texttt{Chorégraphe} au projet. %TODO add Victoria LePin. Avec comme disponibilité: Set your weekly availability to \texttt{Monday and Wednesday and Friday, from 8:00 AM to 2:00 PM}. et Add a vacation period from \texttt{April 27th to May 3rd}.
    \item Modifie le rôle de Victoria pour lui ajouter le rôle de \texttt{Danseur}
\end{enumerate}
\subsection{Ajouter des répétition}
\begin{enumerate}
    \item Ajoute une répétition avec comme nom \texttt{Répétition général}, et qui dure \texttt{5 heures}, avec comme participant \texttt{Jean}, \texttt{Eve} et \texttt{Victoria}.
    \item Ajoute une répétition avec comme nom \texttt{Chorégraphie}, avec une durée de \texttt{3 heures}, avec comme participants \texttt{Jean} et \texttt{Eve}.
    \item Ajoute une répétition avec comme nom \texttt{petite répétition}, qui doit avoir lieu à \texttt{14h} mais pas de date de prévue, avec une durée de \texttt{2 heures},avec comme participant \texttt{Jean}, \texttt{Eve} et \texttt{Victoria}.
\end{enumerate}
\subsection{Ajouter un ordre aux répétitions}
\begin{enumerate}
    \item Ajouter que toutes les répétions doivent avoir lieux avant le répétition général.
\end{enumerate}
\subsection{Génération du calendrier}
\begin{enumerate}
    \item Générer la proposition du calendrier pour le projet \texttt{Aladin}.
    \item Regarder qui est la personne manquante à la répétition \texttt{} %TODO
    \item Accepter l'horaire proposé.
\end{enumerate}

\section{Taches en tant que participant}
\begin{enumerate}
    \item Déconnecter vous
    \item Connectez vous en tant que \texttt{victoria@mail.com} \texttt{1234567}
    \item Regarder les information du projet auquel vous participer, et vos répétitions.
    \item Aller voir votre calendrier
    \item Dite que vous êtes finalement disponible à la répétition \texttt{} %TODO
    \item Modifier vos disponibilité pour indiquer que vous êtes désormais aussi disponible les \texttt{mardi de 14h à 19h}.
\end{enumerate}




\end{document}